\documentclass{article}

\usepackage{nips12submit_e}
\usepackage{times}
\usepackage{url}

\title{Unsupervised morphology induction}

\author{
Victor Chahuneau\\
\texttt{vchahune@cs.cmu.edu}
\And
Phani Gadde\\
\texttt{pgadde@cs.cmu.edu} \\
\And
Peter Schulam\\
\texttt{pschulam@cs.cmu.edu}
}

\begin{document}

\maketitle

\section{Introduction}
\label{sec:introduction}

Building accurate morphological analyzers is a time-consuming task,
which requires linguistic knowledge and abilities to formalize
morphological phenomena into finite-state rules. This approach has
been successful for several European languages, but the majority of
languages still lack such resources. Unsupervised methods are
therefore an interesting alternative that has been extensively
explored and several approaches -- mostly based on
information-theoretic criteria (MDL) -- have been proposed to solve
this problem. Recently, probabilistic models making use of
non-parametric Bayesian techniques have shown competitive performance.

In particular, Goldwater \& al. \cite{goldwater2011} propose a
baseline model for modeling types and tokens in morphological
induction. Lee \& al. \cite{lee2011} suggest an extension which takes
context into account, while Dreyer \& Eisner \cite{dreyer2011} add
structure by encoding morphological phenomena into paradigms that are
tied to grammatical functions, but evaluate their model in a
semi-supervised setting.

Unsupervised morphological analysis is typically done using complex
nonparametric Bayesian models. Deriving the posterior distribution
over quantities of interest (such as the stem lexicon of a language)
can yield complex mathematical expressions which are difficult to
compute. In some applications of Bayesian statistical modeling, it is
acceptable to run samplers for a long time in order to compute a
useful result, but in natural language processing, and, more
specifically, morphological analysis, the amount of data that must be
processed is often large. Furthermore, as morphological analysis is
typically a step used to preprocess tokens in a document before
applying other types of linguistic analysis, it is not

\section{Related Work}
\label{sec:related-work}

\section{Proposed Method}
\label{sec:proposed-method}

\section{Experiments}
\label{sec:experiments}

\section{Conclusion}
\label{sec:conclusion}

\bibliographystyle{plain}
\bibliography{bibliography}

\end{document}
