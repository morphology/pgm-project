Morphology studies the rules that govern the ways that the morphemes
of a language may be put together in meaningful ways. For example,
inflecting the word \textit{walk} to form its present participle form
\textit{walking} indicates that the action is ongoing in the sentence
\textit{I am walking}. In many languages, morphology can play an even
larger role, and may encode many inflectional categories such as
tense, aspect, mood, number, gender, or case. The syntax and semantics
of a sentence are highly dependent on this information, and it is
therefore crucial that natural language processing 

Many natural language processing systems 
