Morphology studies the rules that govern the ways that the morphemes
of a language may be put together in meaningful ways. For example,
inflecting the word \textit{walk} to form its present participle form
\textit{walking} indicates that the action is ongoing in the sentence
\textit{I am walking}. In many languages, morphology can play an even
larger role, and may encode many inflectional categories such as
tense, aspect, mood, number, gender, or case. The syntax and semantics
of a sentence are highly dependent on this information, and it is
therefore crucial that natural language processing systems use
morphological analyses.

Despite the importance of morphology, many state of the art language
technologies such as machine translation, speech recognition, and
question answering do not properly ``parse'' lexical information. This
is in part due to the amount of resources that are necessary to build
proper morphological analyzers. Typical morphological analyzers use
hand-built, finite-state rules to segment words and label each
segmentation with its syntactic purpose. For example, the
\textit{-ing} suffix typically marks the verb as a present participle,
which together with the auxiliary verb \textit{to be} forms the
continuous tense. Compiling such a collection of rules and efficiently
implementing them as a software tool requires expertise in both
linguistics and computer science. Furthermore, such tools must be
rebuilt for each language, which, depending on the availability of
existing corpora and linguistic resources, can be both costly and time
consuming.

There has recently been much interest in automating the process of
morphological analysis using statistical machine learning
algorithms. Automating such systems would address a number of
challenges in natural language processing. First, being able to
automatically compile rich linguistic information for resource-scarce
languages would help to both preserve and better understand languages
with relatively few speakers (compared to the billions that speak
English, for example). Second, automatically creating such resources
could potentially provide an ``free'' way to improve existing language
technologies. For example, \cite{stallard2012} show that an Arabic
machine translation system using an unsupervised morphological
analyzer rivals a system using a supervised analyzer.
