We have proposed a reparametrization of a simple model for modeling morphology that allows its parallel implementation. We have shown empirically its efficiency in terms of speed per iteration, but we introduced a constraint in the model that makes the sampler converge in a larger number of iterations. However, we believe that there is a practical solution to this limiation: use a local collapsed Dirichlet-multinomial in each of the slaves and synchronize the count variations every few iterations to produce a global collapsed base distribution. Although this modification makes the inference inexact, we think this will barely affect the final accuracy of the model.

We also reserve for future work the possibility to make this model of morphology more complex. For example, one could decompose words into a sequence of prefixes, a stem and a sequence of suffixes, which would only require modifying the base distribution. One could also model agreement by using a conditional distribution for generating the prefix, suffix pairs, making the tokens interdependent; this would require some modification in the parallelization scheme but the same independence principles would apply.

We released our implementation of the three models as open-source software (\url{https://github.com/morphology/pgm-project}), offering the first publicly available implementation of such an unsupervised Bayesian model of morphology to our knowledge.
