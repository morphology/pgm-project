\documentclass{article}

\usepackage{sty/final}

\nipsfinalcopy

\title{Faster Unsupervised Morphology Induction}

\author{
Victor Chahuneau\\
\texttt{vchahune@cs.cmu.edu}
\And
Peter Schulam\\
\texttt{pschulam@cs.cmu.edu}
\And
Phani Gadde\\
\texttt{pgadde@cs.cmu.edu} \\
}

\begin{document}

\maketitle

\begin{abstract}
  Morphology studies the rules that govern the ways that the morphemes
  of a language may be put together in meaningful ways. For example,
  inflecting the word \textit{walk} to form its present participle
  form \textit{walking} indicates that the action is ongoing in the
  sentence \textit{I am walking}. In many languages, morphology can
  play an even larger role, and may encode many inflectional
  categories such as tense, aspect, mood, number, gender, or case. The
  syntax and semantics of a sentence are highly dependent on this
  information, and it is therefore crucial that natural language
  processing systems use morphological analyses.  In this work, we
  address the issue of efficiency in Bayesian nonparametric models of
  morphology. Specifically, we formulate and implement a model that
  extends the seminal work of \cite{goldwater2011}. Our new model is
  inspired by a recent reparameterization of DP mixture models that
  allows the Chinese Restaurant Process (CRP) to be split among
  several compute nodes in order to speed up inference
  \cite{williamson2013}. In Section \ref{sec:existing-models}, we
  review the morphological model that we have extended and discuss
  some of the linguistic assumptions encoded in the model. With a
  thorough understanding of the baseline in place, we discuss the DP
  mixture model in Section \ref{sec:dpmm} and review the recent work
  on parallelizing inference. Additionally, we draw connections
  between the baseline model in Section \ref{sec:existing-models} and
  DP mixture models, which motivates the design of our parallelized
  model. In Section \ref{sec:parallel-goldwater} we introduce our
  primary contribution and discuss the decisions made when designing
  and implementing our proposed model. Section \ref{sec:evaluation}
  describes the datasets on which we evaluated our model, and presents
  results demonstrating that our parallel model successfully recovers
  the same analyses as the baseline and significantly improves the
  speed of inference when compared against a baseline serial model.
\end{abstract}

\section{Introduction}
\label{sec:introduction}

Morphology studies the rules that govern the ways that the morphemes
of a language may be put together in meaningful ways. For example,
inflecting the word \textit{walk} to form its present participle form
\textit{walking} indicates that the action is ongoing in the sentence
\textit{I am walking}. In many languages, morphology can play an even
larger role, and may encode many inflectional categories such as
tense, aspect, mood, number, gender, or case. The syntax and semantics
of a sentence are highly dependent on this information, and it is
therefore crucial that natural language processing systems use
morphological analyses.

Despite the importance of morphology, many state of the art language
technologies such as machine translation, speech recognition, and
question answering do not properly ``parse'' lexical information. This
is in part due to the amount of resources that are necessary to build
proper morphological analyzers. Typical morphological analyzers use
hand-built, finite-state rules to segment words and label each
segmentation with its syntactic purpose. For example, the
\textit{-ing} suffix typically marks the verb as a present participle,
which together with the auxiliary verb \textit{to be} forms the
continuous tense. Compiling such a collection of rules and efficiently
implementing them as a software tool requires expertise in both
linguistics and computer science. Furthermore, such tools must be
rebuilt for each language, which, depending on the availability of
existing corpora and linguistic resources, can be both costly and time
consuming.

There has recently been much interest in automating the process of
morphological analysis using statistical machine learning
algorithms. Automating such systems would address a number of
challenges in natural language processing. First, being able to
automatically compile rich linguistic information for resource-scarce
languages would help to both preserve and better understand languages
with relatively few speakers (compared to the billions that speak
English, for example). Second, automatically creating such resources
could potentially provide an ``free'' way to improve existing language
technologies. For example, \cite{stallard2012} show that an Arabic
machine translation system using an unsupervised morphological
analyzer rivals a system using a supervised analyzer.


\section{Existing Models for Unsupervised Morphology}
\label{sec:existing-models}

\begin{figure}[h]
  \centering
  \includegraphics[width=0.6\textwidth]{fig/v1}
  \caption{Baseline model}
  \label{fig:v1}
\end{figure}

\begin{align*}
  \theta_p & \sim \text{Dir}(\alpha_p) \\
  \theta_s & \sim \text{Dir}(\alpha_s) \\
  H(w) & = \sum_{p+s=w} p(p \mid \theta_p) p(s \mid \theta_s) \\
  G & \sim \text{DP}(\alpha, H) \\
  \forall i \in \{1 \dots N\} \\
  w_i & \sim G
\end{align*}


\section{Dirichlet Process Mixture Models}
\label{sec:dpmm}

\begin{figure}[h]
\centering
\includegraphics[width=0.4\textwidth]{fig/v2}
\caption{Reparametrized model}
\label{fig:v1}
\end{figure}

\begin{align*}
\theta_p & \sim \text{Dir}(\alpha_p) \\
\theta_s & \sim \text{Dir}(\alpha_s) \\
H(p, s) & = p(p \mid \theta_p) p(s \mid \theta_s) \\
G & \sim \text{DP}\left(\alpha, H\right)\\
\forall i \in \{1 \dots N\} \\
(p_i, s_i) & \sim G \\
w_i & = p_i+s_i
\end{align*}


\section{Parallelized Goldwater Model}
\label{sec:parallel-goldwater}

\begin{figure}[h]
  \centering
  \includegraphics[width=0.6\textwidth]{fig/v3}
  \caption{Parallelized model}
  \label{fig:v3}
\end{figure}

\begin{align*}
  \theta_p & \sim \text{Dir}(\alpha_p) \\
  \theta_s & \sim \text{Dir}(\alpha_s) \\
  H(p, s) & = p(p \mid \theta_p) p(s \mid \theta_s) \\
  \phi & \sim \text{Dir}\left(\frac{\alpha}{P}\right) \\
  \forall j \in \{1 \dots P\} \\
  G_j & \sim \text{DP}\left(\frac{\alpha}{P}, H\right)\\
  \forall i \in \{1 \dots N\} \\
  \pi_i & \sim \phi \\
  (p_i, s_i) & \sim G_{\pi_i} \\
  w_i & = p_i+s_i
\end{align*}

\subsection{Inference in the Parallel Model}

\section{Evaluation}
\label{sec:evaluation}

\subsection{Datasets}

It is interesting to see how non-parametric models perform on various types of
inputs. Goldwater et. al show the model accuracy and convergeence on English
words. While comparing our models' performace on English with Goldwater et. al,
we also want to look at how the models generalize to languages with similar
morphology. We run the baseline and our models on English verbs (EN-PTB) (the
dataset used in \cite{goldwater2011}) and Russian adjectives (RU-ADJ). The
EN-PTB words are extracted from the Penn Treebank and the RU-ADJ words are from
the Russian national corpus.

The following sections gives a brief overview of the morphology of Ebglish
verbs and Russian adjectives and why they suit the models we are using in this
work.  

\subsubsection{English verbs}

English verbs inflect is very straight forward. Words can be seen as a
combination of a single prefix and a suffix, unlike more morephologically
languages where many morphemes can join to form words. There are mainly three
classes of suffixes that words can take apart from the empty suffix.

\begin{table}[h]
\centering
\begin{tabular}{lll}
\hline
\multicolumn{3}{c}{\textbf{Regular Inflections}} \\
\hline
Morpheme & Description & Examples \\
\hline
-s, -es & occurs wih 3rd person singular & walk-s, run-s, camp-s \\
-ed & past marker & work-ed, bark-ed, camp-ed \\
-n & past participle marker & chose-n, prove-n, woke-n \\
-ing & present participle marker & bark-ing, work-ing, camp-ing \\
\hline
\multicolumn{3}{c}{\textbf{Irregular Inflections}} \\
\hline
\textit{unclear} & past marker & ran, ate \\
\textit{unclear} & past participle marker & drunk, hung \\
\hline
\end{tabular}
\caption{\label{enInflections}Regular and irregular inflections for English verbs}
\end{table}



\subsubsection{Russian adjectives}


\begin{table}[h]
\centering
\begin{tabular}{lcc}
\hline
Dataset & Types & Tokens \\
\hline
EN-PTB & 7K & 113K \\
RU-ADJ & 9K & 18K \\
\hline
\end{tabular}
\caption{Types and Tokens in the datasets}
\label{datastats}
\end{table}

Table~\ref{datastats} gives the types and tokens in the datasets.

\begin{figure}[ht]
\begin{minipage}[b]{0.45\linewidth}
\centering
\includegraphics[width=\textwidth]{fig/en_TokenAccuracies}
\end{minipage}
\hspace{0.5cm}
\begin{minipage}[b]{0.45\linewidth}
\centering
\includegraphics[width=\textwidth]{fig/en_TypeAccuracies}
\end{minipage}
\caption{\label{fig:enacc}Token and Type Accuracies for English verbs over 10K iterations}
\end{figure}

\begin{figure}[ht]
\begin{minipage}[b]{0.45\linewidth}
\centering
\includegraphics[width=\textwidth]{fig/ru_TokenAccuracies}
\end{minipage}
\hspace{0.5cm}
\begin{minipage}[b]{0.45\linewidth}
\centering
\includegraphics[width=\textwidth]{fig/ru_TypeAccuracies}
\end{minipage}
\caption{\label{fig:ruacc}Token and Type Accuracies for Russian adjectives over 110K iterations}
\end{figure}


\begin{figure}[ht]
\begin{minipage}[b]{0.45\linewidth}
\centering
\includegraphics[width=\textwidth]{fig/en_base_lls}
\end{minipage}
\hspace{0.5cm}
\begin{minipage}[b]{0.45\linewidth}
\centering
\includegraphics[width=\textwidth]{fig/ru_base_lls}
\end{minipage}
\caption{\label{fig:basell} Base log-likelihood for all the three models}
\end{figure}

\begin{figure}[ht]
\begin{minipage}[b]{0.45\linewidth}
\centering
\includegraphics[width=\textwidth]{fig/ru_crp_lls}
\end{minipage}
\hspace{0.5cm}
\begin{minipage}[b]{0.45\linewidth}
\centering
\includegraphics[width=\textwidth]{fig/en_crp_lls}
\end{minipage}
\caption{\label{fig:crpll} CRP log-likelihood for all the three models}
\end{figure}


\begin{figure}[ht]
\begin{minipage}[b]{0.45\linewidth}
\centering
\includegraphics[width=\textwidth]{fig/en_lls}
\end{minipage}
\hspace{0.5cm}
\begin{minipage}[b]{0.45\linewidth}
\centering
\includegraphics[width=\textwidth]{fig/ru_lls}
\end{minipage}
\caption{\label{fig:ll} Complete log-likelihood for all the three models}
\end{figure}



\section{Conclusion}
\label{sec:conclusion}

Future work (things we will never do but someone should really think about):
- actually using tokens to do something useful (see the litterature examples given before)
- finding a practical way to do collapsed sampling of the base

TODO link to code


\bibliographystyle{plain}
\bibliography{bib/final}

\end{document}
